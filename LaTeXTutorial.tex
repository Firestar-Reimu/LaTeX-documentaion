%! Author = saili
%! Date = 2019/8/23/0023

% Preamble
\documentclass{article}
\input{preamble/packages.tex}
\input{preamble/color.tex}
\input{preamble/globalpre.tex}
\input{preamble/localpre.tex}
\input{preamble/format.tex}
\input{preamble/box-support.tex}
\newtcbox{\highlight}[1][yellow]{on line,
% sharp corners=south,
arc=0mm,outer arc=0mm,colback=#1!10!white,colframe=#1!50!black,
boxsep=0pt,left=1pt,right=1pt,top=1pt,bottom=1pt,
boxrule=-0.1pt,bottomrule=1pt}

\newtcbox{\highunderline}[1][red]{
    on line,
    % sharp corners=south,
    arc=0mm,outer arc=0mm, colback=white,colframe=#1!50!black,
    boxsep=0pt,left=1pt,right=1pt,top=1pt,bottom=1pt,
    boxrule=-0.1pt,bottomrule=1pt
}


\bibliographystyle{plain}

\usepackage[printwatermark,disablegeometry]{xwatermark}
\newwatermark[page=1,color=gray!20,angle=45,scale=4,xpos=0,ypos=0]{SAILIST}
\newwatermark[pages=2-4,color=gray!20,angle=45,scale=3,xpos=0,ypos=0]{\LaTeX{}速查手册}
\newwatermark[pagex={5,6,7},color=gray!20,angle=45,scale=3,xpos=0,ypos=0]{\LaTeX{}速查手册}
% %信息
\title{\LaTeX{}速查手册}
\author{Sailist}

% \newcommand*{\refname}{Bibliography}

\begin{document}
    \bibliography{ref}
    \maketitle
    \clearpage
    \renewcommand{\baselinestretch}{0.75}\normalsize
    \tableofcontents
    \listoffigures
    \listoftables
    \newpage
    \renewcommand{\baselinestretch}{1.3}\normalsize
    \input{contents/install.tex} % 安装部署
    \input{contents/pages.tex} %页面综述
    \input{contents/distance.tex} % 大小与间距
    \input{contents/sections.tex} % 标题和目录
    \input{contents/text.tex} %普通文本
    \input{contents/itemize.tex} %列表
    \input{contents/tabulars.tex} %表格
    \input{contents/figure.tex} %图片
    \input{contents/equation.tex} %公式环境
    \input{contents/floats.tex} %浮动窗口
    \input{contents/notes.tex} %脚注、边注、参考文献
    \input{contents/colour} % 颜色
    
    \input{contents/commandandenviron.tex}

    \section{代码环境方案}
    \subsection{程序代码}
    \subsubsection{抄录环境}
    \subsubsection{listing}
    \subsubsection{minted}
    \subsubsection{最佳实践:tcolorbox}
    \subsection{伪代码}
    
    % https://blog.csdn.net/simple_the_best/article/details/52710794 伪代码环境
    \subsection{树结构}
    % Latex树状结构-Forest http://softlab.sdut.edu.cn/blog/xuqianhui/2017/06/28/latex%e6%a0%91%e7%8a%b6%e7%bb%93%e6%9e%84-forest/
    \section{编写结构}
    
    \todo{引入文件/引用问题/编译参数/目录/目录问题/代码风格/abspath}
    
    \todo{更改目录结构后,对各种库、bibtex的影响和解决}

    \section{库使用}
    \subsection{minipage}\label{sub:minipage}
    minipage宏包用于在一页中创建一个小的页面(实质是一个盒子),也具有很强的可定制性。
    \subsection{tcolorbox}
    本手册使用的全部带颜色的边框均为tcolorbox的实现,其基本的使用非常简单,在熟悉\LaTeX{}的基本操作后,可以查看tcolorbox的宏包说明手册来完成,其所有的边框,间隔,圆角,文字,分割线都可以自定义,同时还对代码的显示进行了封装,可以更自由的定制代码的显示风格。

    \subsection{adjustbox}
    是一个用于任意调整“盒子”位置的宏包,盒子是\LaTeX{}中的基本单位,所以该宏包在一些场合的定制性非常的强。
    
    \section{高级使用}
    \todo{有生之年系列:文档类撰写、Latex语言深入、库撰写、Tex编译结构}
    \todo{MakeFile编写}

    \section{模板收录}
    国赛模板:\href{https://github.com/latexstudio/CUMCMThesis}{Github链接}
    
    \todo{简历模板、PPT模板、笔记模板、书模板...}


\end{document}
% 
